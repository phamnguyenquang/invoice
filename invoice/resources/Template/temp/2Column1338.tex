\documentclass[a4paper]{letter}
\usepackage{fp}
\usepackage{booktabs}
\usepackage{ragged2e}
\usepackage{longtable}
\usepackage[T1]{fontenc}
\usepackage[utf8]{inputenc}
\usepackage{textcomp}
\usepackage{eurosym}
\newcounter{cnt}
\setcounter{cnt}{0}
\def\inc{\stepcounter{cnt}\thecnt}
\gdef\TotalHT{0}
\newcommand{\product}[4]{%
\inc &#1  &#2   &#3  &#4   &\FPmul\temp{#3}{#4}\FPround\temp{\temp}{3}\temp 
%% Totalize
\FPadd\total{\TotalHT}{\temp}%
\FPround\total{\total}{3}%
\global\let\TotalHT\total%
\\ }
\newcommand{\totalttc}{
   \TotalHT  }
\begin{document}
\RaggedRight
%SENDER INFORMATION
%---------------------------------------
%
\begin{minipage}[t]{0.5\textwidth}
	\begin{flushleft}

	\end{flushleft}
\end{minipage}%
\begin{minipage}[t]{0.5\textwidth}
 	\begin{flushright}
		\textbf{\huge  Aliyah Bergqvist Pte Ltd} \\
		{\large Eden 84}\\
		{\large 81918, Minsk, Bangkok, Turkey }\\
		{\large email@example.net}
	
 	\end{flushright}
\end{minipage}%

\begin{minipage}[t]{0.5\textwidth}
	\begin{flushleft}

	\end{flushleft}
\end{minipage}%
\begin{minipage}[t]{0.5\textwidth}
 	\begin{flushright}
		\textbf{\large Date:} 18 April 2020\\		
 	\end{flushright}
\end{minipage}%
%---------------------------------------

%RECEIVER INFORMATION
%---------------------------------------
\begin{minipage}[t]{0.5\textwidth}
	\begin{flushleft}
	%---------------------------------------
	 	\textbf{\large Jonathan Blom } \\
		{\large Gabriel 67}\\
		{\large 75681, Beijing, Chiang Mai, Argentina }\\
	%---------------------------------------
	\end{flushleft}
\end{minipage}
\begin{minipage}[t]{0.5\textwidth}
 	\begin{flushright} 	
 	%---------------------------------------
	
		%---------------------------------------
 	\end{flushright}
\end{minipage}%

\begin{minipage}[t]{0.5\textwidth}
	\begin{flushleft}
	%---------------------------------------
		\textbf{\large Invoice Number:} 231001\\
	%---------------------------------------
	\end{flushleft}
\end{minipage}
\begin{minipage}[t]{0.5\textwidth}
 	\begin{flushright} 	
 	%---------------------------------------
	
		%---------------------------------------
 	\end{flushright}
\end{minipage}%
%
%
%---------------------------------------
\begin{longtable}{cp{4.0cm}rrrr}
\toprule
Item   &Description &  &Price  & Qty & Total \\
\midrule
    \product{Computer peripherals}{€}{1000.00}{1}
    \product{Harddisk 2000E}{€}{2000}{1}
    \product{The \TeX book}{€}{100.00}{100}
    \product{Mouse Four}{€}{5000.00}{1}
    \product{Keyboard Five}{€}{5000.00}{2}
\midrule
    &&&&& Total \totalttc\\
\bottomrule
\end{longtable}
\end{document}